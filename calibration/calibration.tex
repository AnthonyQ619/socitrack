\documentclass{article}

\usepackage[top=.75in, bottom=.75in]{geometry}

\usepackage{amsmath}
\usepackage{xcolor}

\begin{document}

\newcommand{\tda}[1]{\textcolor{red}{#1}}
\newcommand{\tdb}[1]{\textcolor{orange}{#1}}
\newcommand{\tdc}[1]{\textcolor{blue}{#1}}
\newcommand{\unk}[1]{\textcolor{red}{#1}}
\newcommand{\der}[1]{\textcolor{orange}{#1}}
\newcommand{\off}[2]{\textnormal{\emph{Off}}_{#1\rightarrow#2}}

\textbf{Setup:} Assume 3 polypoint tags, all 1~m from each other.

\begin{itemize}
  \item t = tx delay calibration factor (unknown)
  \item r = rx delay calibration factor (unknown)
  \item $\lambda$ = time of flight for 1~m ($\lambda = \frac{1}{c}\times\textnormal{\texttt{DWT\_TIME\_UNITS}}$)
  \item $\epsilon$ = precise delay before sending the next packet (i.e.\ 5\,ms)
\end{itemize}

$A:$ Tag 0 sends a packet
\begin{align}
  ARX1 &= ATX0 + t_0 + r_1 \\
  ARX2 &= ATX0 + t_0 + r_2 \\
  ARX3 &= ATX0 + t_0 + r_3
\end{align}

$B:$ Tag 0 sends another packet
\begin{align}
  BRX1 &= BTX0 + t_0 + r_1 \\
  BRX2 &= BTX0 + t_0 + r_2 \\
  BRX3 &= BTX0 + t_0 + r_3
\end{align}

$C:$ Tag 1 sends a packet such that $CTX1 = BRX1 + \epsilon$\,ms
\begin{align}
  CRX0 &= CTX1 + t_1 + r_0 \\
  CRX2 &= CTX1 + t_1 + r_2 \\
  CRX3 &= CTX1 + t_1 + r_3
\end{align}

$D:$ Tag 2 sends a packet such that $DTX2 = CRX2 + \epsilon$\,ms
\begin{align}
  DRX0 &= DTX2 + t_2 + r_0 \\
  DRX1 &= DTX2 + t_2 + r_1 \\
  DRX3 &= DTX2 + t_2 + r_3
\end{align}

$E:$ Tag 3 sends a packet such that $ETX3 = DRX3 + \epsilon$\,ms
\begin{align}
  ERX0 &= ETX3 + t_3 + r_0 \\
  ERX1 &= ETX3 + t_3 + r_1 \\
  ERX2 &= ETX3 + t_3 + r_2
\end{align}

Note that while all of [ABCDE][RT]X[0123] are known values, they cannot
be used directly since they are measured by different clocks. To convert ARX1
to ARX0, we need to derive a correction factor $k$ for the difference in
oscillator speed between each node and find the offset of each clock \emph{at
that point in time}.  Care must be taken to ensure that $k$ is factored in
consistenly throughout the calibration.

We make the assumption that each $k$ remains constant throughout the
calibration procedure. We use the first two packets $A$ and $B$ to find $k$
values for the rest of the round:
\[ k_{0\rightarrow1} = \frac{BRX1-ARX1}{BTX0-ATX0} ,%
   k_{0\rightarrow2} = \frac{BRX2-ARX2}{BTX0-ATX0} ,%
   k_{0\rightarrow3} = \frac{BRX3-ARX3}{BTX0-ATX0} \]

\[ k_{1\rightarrow2} = \frac{BRX2-ARX2}{BRX1-ARX1} ,%
   k_{1\rightarrow3} = \frac{BRX3-ARX3}{BRX1-ARX1} ,%
   k_{2\rightarrow3} = \frac{BRX3-ARX3}{BRX2-ARX2} \]

\[
  \textnormal{\emph{Other k's:}}~~~
  k_{1\rightarrow0} = \frac{1}{k_{0\rightarrow1}},~~~
  k_{2\rightarrow0} = k_{2\rightarrow1} \times k_{1\rightarrow0},~~~
  \textnormal{\emph{etc}}
\]

\newpage

Now we set up equations that exploit the known ranges between tags.
$\lambda_{01}$ is the distance between tags 0 and 1 in decawave time units.
If $t_0 = 0$, $r_1 = 0$, $k_{1\rightarrow0} = 1$, and $B\off{1}{0}0 = 0$, that is
there is no transmission delay, no receive delay, and perfect clock
calibration, then
\[
  BRX1 \times k_{1\rightarrow0} - B\off{1}{0}0 = BRX1 = BRX0
  ~~~\textnormal{and}~~~
  BRX0 - BTX0 = \lambda_{01}
\]
In practice, $t_0 \neq 0$ and $r_1 \neq 0$. We define the term $\Delta_{0,1}$
to express the difference between the expected value and the measured value:
\[
  BRX0 - BTX0 - \lambda_{01} = \Delta_{0,1} = t_0 + r_1
\]
Critically, we do not assume that $t_0 = r_0$, that is the transmit and
receive delay of a single node may not be symmetric. This means than
$\Delta_{0,1}$ is not necessarily equal to $\Delta_{1,0}$.

\medskip
\noindent
With the above in mind, the following equations provide sufficient relations.
We convert everything to tag 0's time domain to make computing the offsets correctly easier.
For clarity, in this section we color variables if the are \unk{unknown until
the end} and \der{trivially derived from other unknowns}.

\begin{align}
BRX0 \times k_{1\rightarrow0} - \unk{B\off{1}{0}0} - BTX0 - \lambda_{01} &= \Delta_{0,1} = \unk{t_0} + \unk{r_1} \\
BRX0 \times k_{2\rightarrow0} - \unk{B\off{2}{0}0} - BTX0 - \lambda_{02} &= \Delta_{0,2} = \unk{t_0} + \unk{r_2} \\
BRX0 \times k_{3\rightarrow0} - \unk{B\off{3}{0}0} - BTX0 - \lambda_{03} &= \Delta_{0,3} = \unk{t_0} + \unk{r_3} \\
~\nonumber\\
CRX0                                                 - (CTX1 \times k_{1\rightarrow0} - \der{C\off{1}{0}0}) - \lambda_{01} &= \Delta_{1,0} = \unk{t_1} + \unk{r_0} \\
(CRX2 \times k_{2\rightarrow0} - \der{C\off{2}{0}0}) - (CTX1 \times k_{1\rightarrow0} - \der{C\off{1}{0}0}) - \lambda_{12} &= \Delta_{1,2} = \unk{t_1} + \unk{r_2} \\
(CRX3 \times k_{3\rightarrow0} - \der{C\off{3}{0}0}) - (CTX1 \times k_{1\rightarrow0} - \der{C\off{1}{0}0}) - \lambda_{13} &= \Delta_{1,3} = \unk{t_1} + \unk{r_3} \\
~\nonumber\\
DRX0                                                 - (DTX2 \times k_{2\rightarrow0} - \der{D\off{2}{0}0}) - \lambda_{02} &= \Delta_{2,0} = \unk{t_2} + \unk{r_0} \\
(DRX1 \times k_{1\rightarrow0} - \der{D\off{1}{0}0}) - (DTX2 \times k_{2\rightarrow0} - \der{D\off{2}{0}0}) - \lambda_{12} &= \Delta_{2,1} = \unk{t_2} + \unk{r_1} \\
(DRX3 \times k_{3\rightarrow0} - \der{D\off{3}{0}0}) - (DTX2 \times k_{2\rightarrow0} - \der{D\off{2}{0}0}) - \lambda_{23} &= \Delta_{2,3} = \unk{t_2} + \unk{r_3} \\
~\nonumber\\
ERX0                                                 - (ETX3 \times k_{3\rightarrow0} - \der{E\off{3}{0}0}) - \lambda_{03} &= \Delta_{3,0} = \unk{t_3} + \unk{r_0} \\
(ERX1 \times k_{1\rightarrow0} - \der{E\off{1}{0}0}) - (ETX3 \times k_{3\rightarrow0} - \der{E\off{3}{0}0}) - \lambda_{13} &= \Delta_{3,1} = \unk{t_3} + \unk{r_1} \\
(ERX2 \times k_{2\rightarrow0} - \der{E\off{2}{0}0}) - (ETX3 \times k_{3\rightarrow0} - \der{E\off{3}{0}0}) - \lambda_{23} &= \Delta_{3,2} = \unk{t_3} + \unk{r_2}
\end{align}

Leveraging the precise $\epsilon$ between packets (i.e.\ $CTX1 = BRX1 + \epsilon$):
\begin{align}
\der{C\off{1}{0}0} &= \unk{B\off{1}{0}0} + \epsilon \times k_{1\rightarrow0} \\
\der{C\off{2}{0}0} &= \unk{B\off{2}{0}0} + \epsilon \times k_{1\rightarrow0} \\
\der{C\off{3}{0}0} &= \unk{B\off{3}{0}0} + \epsilon \times k_{1\rightarrow0} \\
~\nonumber\\
\der{D\off{1}{0}0} &= \der{C\off{1}{0}0} + \epsilon \times k_{2\rightarrow0} \\
\der{D\off{2}{0}0} &= \der{C\off{2}{0}0} + \epsilon \times k_{2\rightarrow0} \\
\der{D\off{3}{0}0} &= \der{C\off{3}{0}0} + \epsilon \times k_{2\rightarrow0} \\
~\nonumber\\
\der{E\off{1}{0}0} &= \der{D\off{1}{0}0} + \epsilon \times k_{3\rightarrow0} \\
\der{E\off{2}{0}0} &= \der{D\off{2}{0}0} + \epsilon \times k_{3\rightarrow0} \\
\der{E\off{3}{0}0} &= \der{D\off{3}{0}0} + \epsilon \times k_{3\rightarrow0}
\end{align}


We have 12 knowns $\Delta_{0,1} \dots \Delta_{3,2}$ for 11 unknowns
$\unk{t_0, r_0, t_1, r_1, t_2, r_2, t_3, r_3, B\off{1}{0}0, B\off{2}{0}0, B\off{3}{0}0}$.

\medskip
\noindent
\emph{Note if we had only used three anchors, we would have only 6 knowns
$\Delta_{0,1} \dots \Delta_{2,1}$ for 8 unknowns. The fouth anchor adds 6 more
knowns at the cost of $r_3, t_3, \textnormal{and}~B\off{3}{0}$.}


\end{document}
